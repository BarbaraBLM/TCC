The number of security cameras in cities has been increasing, not only in public places, but even on a personal level, many times we have cameras in our homes. They are used so that we feel more safe and so that, in the case of an incident, the appropriate actions can be taken as quickly as possible. However, with this increase in the number of cameras, it becomes humanly impossible to keep watching the images to become aware of any incident that may have occurred. It is in this context that technology can help, through \textit{Machine Learning} algorithms it is possible that anomalous events are detected automatically.

Studies about anomaly detection using images captured by cameras have been done for some years, usually using convolutional neural networks. The goal of this work, however, is to add \textit{captions}, that is, images descriptions, in addition to the images themselves, and to verify if there is any expressive gain in the algorithm's accuracy using the \textit{captions}.





%Cada vez mais tem crescido o número de câmeras de segurança presentes nas cidades, não só em locais públicos, mas até a nível pessoal, muitas vezes possuímos câmeras em nossas casas. Elas são usadas para que tenhamos mais seurança e para que, no caso de algum incidente, as atitudes cabíveis possam ser tomadas o mais rápido possível. Porém, com esse aumento do número de câmeras, torna-se humanamente impossível ficar observando as imagens para tomar ciência de qualquer incidente que possa ter ocorrido. É nesse contexto que a tecnologia pode ajudar, através de algoritmos de \textit{Machine Learning} é possível que os eventos anômalos sejam detectados automaticamente. 

%Estudos sobre detecção de anomalia utilizando as imagens capturadas pelas câmeras já vem sendo feitos há alguns anos, geralmente utilizando redes neurais convolucionais. O objetivo desse trabalho no entanto, é adicionar \textit{captions}, ou seja, descrições das imagens, além das imagens em si, e verificar se há algum ganho expressivo no acerto do algoritmo utilizando as \textit{captions}.

%Será exatamente a tradução para o inglês do resumo.%