Cada vez mais tem crescido o número de câmeras de segurança presentes nas cidades, não só em locais públicos, mas até a nível pessoal, muitas vezes possuímos câmeras em nossas casas. Elas são usadas para que tenhamos mais seurança e para que, no caso de algum incidente, as atitudes cabíveis possam ser tomadas o mais rápido possível. Porém, com esse aumento do número de câmeras, torna-se humanamente impossível ficar observando as imagens para tomar ciência de qualquer incidente que possa ter ocorrido. É nesse contexto que a tecnologia pode ajudar, através de algoritmos de \textit{Machine Learning} é possível que os eventos anômalos sejam detectados automaticamente. 

Estudos sobre detecção de anomalia utilizando as imagens capturadas pelas câmeras já vem sendo feitos há alguns anos, geralmente utilizando redes neurais convolucionais. O objetivo desse trabalho no entanto, é adicionar \textit{captions}, ou seja, descrições das imagens, além das imagens em si, e verificar se há algum ganho expressivo no acerto do algoritmo utilizando as \textit{captions}.

%Essa parte será o resumo do trabalho em poucas frases. Nessa parte deverá conter uma explicação sobre o problema e a proposta, finalizando em como ela se comportou com relação as demais soluções da literatura.