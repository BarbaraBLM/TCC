\chapter{Experimentos}\label{chp:EXPERIMENTOS}

Este capítulo falará da solução em execução, ou seja, quais ferramentas escolhidas e seus motivos, como ele foi desenvolvido e como ele atuou em comparação aos trabalhos relacionados.

\section{Base de Dados}

% Falar sobre a escolha de um arquivo, captação de frames, divindo por 3, cerca de 1000 imagens %

\section{Metodologia}
\subsection{OFA}
\subsection{O \textit{framework} Gensim}
\subsection{O \textit{framework} Pytorch}
\subsection{O \textit{framework} ScikitLearn}
\subsection{Validação dos Modelos}
% Flr da divisão dos dados em 3 conjuntos: treino, teste e validação %  
\subsection{Métricas}
\subsection{Arquitetura da Rede Neural Convolucional e Multilayer Perceptron}
\subsection{Arquitetura da Rede Neural Convolucional}
\subsection{Escolha do Limiar de Anomalia}

\section{Resultados}
% 1. tabela com as informações das execuções (quantas forem, as 20, 30, 40, execuções q eu fizer, sei lá)
%  Informações: erro médio para cada classe, f1 score e acurácia para cada modelo. //nem sei se precisa de standard deviation 
 
% 2. tabela com a média das informações da tab anterior (resumo dos resultados) //acho q vai ser tipo uma cópia da tabela lá de cima do capítulo 1, resumo dos resultados

% 3. matrizes de confusão de cada modelo

% 4. É aqui mesmo que vc vai falar da conclusão do experimento, se colocar as captions ajudou ou não na detecção das anomalias. Se sim ou se não, dissertar possíveis motivos.

%%%%%%%%%%%%%%%%%%%%%%%%%%%%%%%%%%%%%%%%%%%%%%%%%%%%%%%%%%%%%%%%%%%%%%%%%
% \section{Métrica de Avaliação}

% \section{Análise dos Experimentos}

% \subsection{Mixed-input}
% \subsection{Classificador Mixed-input}
% \subsection{Convolutional Neural Network}
% \subsection{Classificador Convolutional Neural Network}
% \subsection{Experimento Mixed-input}
% \subsection{Experimento Convolutional Neural Network}

% \section{Resultados}
% \subsection{Experimento Mixed-input}
% \subsection{Experimento Mixed-input}    