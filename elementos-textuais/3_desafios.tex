\chapter{Desafios da Detecção de Anomalia }\label{chp:DESAFIOS_DETECCAO_DE_ANOMALIA}

Esse capítulo tem como objetivo promover um maior entendimento sobre o tema Detecção de Anomalia, tratando-o com mais detalhes e explicando suas motivações e desafios. Será abordado também o tema Aprendizado de Máquina, primordial para o entendimento do problema, e seus diversos tipos. E, por último, Redes Neurais, que geralmente é a técnica de aprendizado de máquina utiliza para a identificação de cenários anômalos.

% \section{Detecção de Anomalia}
% O reconhecimento de padrões é muito importante para diversas áreas da ciência. Os objetos a serem classificados podem ser vários, como imagens, formas de onda para aplicação em speech recognition ou produtos em uma fábrica. Hoje em dia, o reconhecimento de padrões é uma parte primordial de vários sistemas de \textit{machine learning} voltados para tomada de decisões.

% \subsection{Desafios da Detecção de Anomalia}
% % dificuldade e achar dataset labeled, desbalanceamento de classe, fortemente dependente do contexto %
% \subsection{Métodos para Detecção de Anomalia}
% %só falar q um classificador não rola devido ao desbalanceamento, q podemos simplesmente treinar %
\section{Desbalanceamento de Classe}

\section{Forte Dependência do Contexto}

\section{Dificuldade em Encontrar Dados Rotulados}

\section{Modelo Utilizado para Detecção de Anomalia}

\begin{comment}
Esse capítulo será responsável por dar uma base teórica a proposta. 
A ideia é que o leitor, com um breve conhecimento prévio, tenha a capacidade de ler esse material e consiga ter a capacidade de entender tecnicamente a proposta do projeto final.
Dependendo do assunto, poderá ter mais de uma área de resumo, podendo ficar cada uma em uma seção ou serem capítulos separados.

No caso de um único capítulo, deverá ter um breve resumo antes de iniciar a seção explicando o que este capítulo fará. Caso seja capítulo separado, poderá introduzir a área diretamente sem a criação de uma seção ou criar uma seção chamada "Introdução" e o nome do capítulo pode ser o nome da área.
\end{comment}