\chapter[Introdução]{Introdução}

Contextualize o seu trabalho de forma sucinta. Delimite o seu tema de estudo. Convença o leitor da relevância e importância do seu trabalho.
 

\section{Motivação}
Introduza o leitor ao assunto, descreva os fatores motivadores para o desenvolvimento do seu trabalho.   Descreva brevemente o estado da arte e indique os problemas que ainda não foram resolvidos. Faça um gancho para a próxima sub-seção em que você descreve os  objetivos do seu trabalho. 

\section{Objetivos e Desafios da Pesquisa}
Descreva claramente os desafios que o tema propõe e quais os  objetivos que se pretende alcançar. Se o tema for muito abrangente, descreva os objetivos em termos de "objetivo geral" e  "objetivos específicos". Cuidado com objetivos como "desenvolver um sistema...."; "explorar um método ...".Esses objetivos são triviais, ou seja, uma vez desenvolvido o sistema ou explorado o método, independente dos resultados, o objetivo foi atingido. Prefira verbos como: "contribuir", "analisar", "investigar", "comparar". Os membros da banca ao lerem essa seção farão o seguinte questionamento: Algum conhecimento novo para a humanidade foi produzido?


\section{Hipótese}
Descreva claramente quais são as hipóteses da sua pesquisa (Uma hipótese é uma suposição para a solução do problema que você pretende desenvolver). Indique quais perguntas estão associadas a sua hipótese. Lembre-se que as hipóteses deverão ser comprovadas via os experimentos que serão descritos no capítulo \ref{experimentos}.

\section{Contribuições}
Liste as contribuições do seu trabalho. Lembre-se que publicações não são contribuições científicas do seu trabalho. Haverá uma seção específica com esse fim.

\section{Organização da Dissertação ou Tese}
Descreva como a sua monografia está organizada, descrevendo brevemente o conteúdo de cada capítulo.